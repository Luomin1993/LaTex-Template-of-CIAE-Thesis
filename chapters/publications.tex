\begin{publications}

\section*{已发表论文}

\begin{enumerate}
\item Luo M Z . Automatic Derivation of Formulas by Graph Embedding and Pattern Matching Network[J]. ACM.2019.(icone2018录用)(第一作者);
\item Luo M Z , He X F , Liu L . Generative Model for Material Experiments Based on Prior Knowledge and Attention Mechanism[C]. 2018.(IEEE/ICSESS收录)(第一作者);
\item Luo M Z , Liu L . Nuclide Recognizition Using Online Learning[C]. 2018.(ACM/SAC收录)(第一作者);
\item Luo M Z , Tong Y , Liu J C . Orthogonal Policy Gradient and Autonomous Driving Application[J]. IEEE.2018.(NIPS2018 Workshop收录)(第一作者);
\end{enumerate}

\section*{待发表论文预印本}

\begin{enumerate}
\item Luo M Z* , Yu S* .Visual Semantic Reasoning: from Concept Learning to Logic Inference[A]. 2018.(第一作者);
\item Luo M Z .Gibbs Model for Automatic Trading[A]. 2018.(第一作者);
\item Luo M Z .Conceptual Aggregation Graphs CAG for Natural Language Representation and Reasoning[A]. 2018.(第一作者);
\item Luo M Z .MEM-NN: A Classification Network Based on Maximum Entropy Method[A]. 2018.(第一作者);
\end{enumerate}

\section*{研究报告}
\begin{enumerate}
\item 猕猴视神经实验的双光子成像与智能控制系统的构建(2019年2月)
\item 使用二阶梯度作正则项交叉训练参数(2019年1月)
\item 使用SGHMC(随机梯度哈密顿-蒙特卡罗方法)做姿态估计(2018年12月)
\item 正交策略梯度法和自动驾驶应用(2018年12月)
\item 在Keras中如何按最大似然(Max Likewood)训练模型(2018年12月)
\item 具有高层控制模块的自编码器(2018年12月)
\item 基于Fuzzy World生成的Concept Learning数据集(2018年12月)
\item 符号的概念学习:超越主流表征学习的学习方法(2018年10月)
\item 关于量化交易的物理模型的一种新方法及思考(2018年9月)
\item 用于3D虚拟交互式强化学习环境构建的小工具Fuzzy World(2018年9月)
\item 从辐照实验条件到材料辐照肿胀图像:运用变分自编码器VAE和生成式对抗网络WGAN-gp(2018年7月)
\item 材料辐照图片气泡计数的算法和程序(2018年4月)
\item 智云视图中文车牌识别源码解析系列(2018年2月)
\item 深度学习之Caffe完全掌握系列(2017年12月)
\item 粒子物理和消息队列库系列(2017年11月)
\item 粒子物理蒙特卡罗模拟库Geant4开发系列(2017年10月)


\end{enumerate}

\end{publications}
