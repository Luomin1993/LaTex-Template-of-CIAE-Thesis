
%%% Local Variables:
%%% mode: latex
%%% TeX-master: t
%%% End:

\chapter{总结及展望}
\label{cha:intro}

如何将人类的学习和推理能力赋予机器一直以来都是人工智能研究的热点方向 \upcite{Han1969QA} \upcite{Banerjee1969Reactive} \upcite{Lin1992Self}。在本文中,根据生物实验研究的启发,再结合对当前主流深度学习方法依赖的梯度下降算法的缺陷的分析,提出了基于类额叶编码的学习和推理模型,简称智能学习推理系统,


...

\section{后续工作的展望}
机器智能学习推理能力的构建是一项极具挑战性的研究课题,本论文仅针对其中的几个关键问题开展了研究,但即使针对这几个问题,也还有很多工作需要进一步深入,主要包括: 

\subsection{关于清醒猴的双光子成像实验的探究}
虽然本文涉及的实验很好地给出了生物关于表征客体和基于反馈机制优化自身的结论,但是实验的设计仍然可以深入优化,使得可以探究关于猕猴更高级认知推理能力执行时,其额叶皮层的活动规律。从实验设置上来说,这还需要引入更复杂的屏幕视觉信号引导程序,以及对猕猴外科手术和诸多排除手术后创口恢复更复杂的技术(因为额叶在灵长类动物的脑前额部位,手术难度更高),但是如果能实现对更高级认知活动的观测,进而就可以辅助构建更高效的智能学习推理系统,这将是本文后续的研究工作方向。

\subsection{智能学习推理系统的数学模型的理论性}
在本文前部分,从人工智能推理模型、发展历史、主要技术方法归类、若干开发问题和研究状况进行了总结,并指出了当前智能学习推理模型的缺陷和改进思路。基于汲取和认识之前模型的精髓和缺陷,本文构建出了更具鲁棒性和泛化能力的类额叶智能学习模型,并且引入了实分析和泛函分析的理论,在比较强的数学范式下讨论了模型的收敛性和稳定性问题。但是模型的数学理论仍然存在一些漏洞,比如模型误差的收敛上下界的可逼近性已经证明,但是并没有给出具体收敛上下界的形式,也没有基于可积性讨论单个神经编码元的机制,这些都是本文接下来需要进一步完善的工作。在完善后,智能学习推理系统的机制理论会尽可能地成为一个自封闭的数学子系统。


\subsection{模型的计算机编程实现}
基于模型的设计和在数理逻辑、工业控制、文学理解的具体应用,本文在GNU/Linux下利用Python和Keras库 \upcite{chollet2015keras} 开发了计算机程序来检验模型的可行性。目前三个程序是分拆的,即针对不同的应用场景,需要调用不同的程序来完成任务,这在一定程度上和类脑智能的通用性仍然存在矛盾,下一步,将在模型对客体的认知范式上做优化,目标是使程序可以通过公共的接口(Application Programming Interface,API)实现统一的自学习模式,进而成为和具体场景无关的泛化学习效果更佳的智能应用程序。欲实现这样的功能,需要在数据集的预处理环节上增加预处理程序模块,这将是本文下一步的研究内容。


最后,需要进一步指出的是,尽管智能学习推理并非一个新的研究课题,但仍然是一项充满挑战问题的研究课题 \upcite{Hecht1987Counterpropagation} \upcite{Newell1956The},需要投入大量精力对其进行更深入地研究,尤其需要深入研究人类和动物的认知学习能力的神经机制,也需要就智能学习推理的数学建模及理论开展系统研究,需要借鉴计算机科学、现代控制科学、语言学、认知心理学等领域的最新进展 \upcite{Pearl1988Probabilistic}。相信经过研究人员深入细致的研究,自动智能学习推理问题最终一定能够得到更优的解决并造福人类文明!