
%%% Local Variables:
%%% mode: latex
%%% TeX-master: t
%%% End:

\chapter{基于脑额叶皮层启发的神经编码模型}
\label{cha:intro}

在本章中将承接上一章中从清醒猕猴脑视觉皮层的双光子成像实验和其他关于脑额叶认知能力的研究 \upcite{Paus2001Primate} \upcite{Valois1982Spatial} \upcite{Rhodes2010Apical} 得到的关于智能认知学习能力的启发,开始引入数学理论工具构建本文的智能学习推理系统。首先我们会结合实分析和泛函分析的数学内容扩展加强在上一章中初步提出的智能学习推理的数学范式,形成一个数学模型雏形。接着会关联即将提出的详细的智能学习推理模型,给出必须的数学理论基础,这一部分是至关重要的,因为本文不会引入冗余的数学理论,而提及的数学理论基础是和构建智能学习推理系统密切相关的,并且相当于对客观世界的一个基础的数学世界观(可视化的描述可见图~\ref{fig:fig_20})。在引入数学理论基础之上,将类比大脑的前额叶认知功能,提出本文的连续皮层介质渗流编码模型,这个命名比较直观地概括了本文编码模型的特点,和传统神经网络不同的是,本文中模型神经元的输出是一个沿输出皮层的连续膜电压信号而非离散信号,并且本文并未采用基于梯度下降的优化算法;同时,关于所提出模型优化方法的平稳性的数学证明也被给出。最后,进一步地,本文结合所提出模型的数学理论,给出了一些关于实分析和泛函分析的通用的有意义的数学结论,这些结论不光适用于本文的神经编码模型,其作为纯数学理论也成立并有一定的推广价值。本章中给出的神经编码模型对后续构建完整的学习推理能力有重要的支撑作用。


  
\section{脑皮层表征的客观世界的数学设定}
在这一小节里将给出的是智能学习推理系统的基础:


...



\begin{thm}
\textbf{(由神经元编码函数的被编码点集)}证明Lusin定理的逆定理一个方法;
\end{thm}

\begin{proof}
Lusin定理的逆定理可以表述为:$f(x)$是$E$上的函数,对任意$\delta>0$,存在闭子集$E_{\delta} \subset E$使得$f(x)$在$E_{\delta}$上是连续函数,且$m(E-E_{\delta})<{\delta}$,则$f(x)$是$E$上$a.e.$(几乎处处成立)的有限可测函数。

证:对任意$1/n$,存在闭子集$E_{n} \subset E$,使得$f(x)$在$E_n$上连续,且:

$$m\left(E-E_{n}\right)<\frac{1}{n}$$

令$E_{0}=E-\bigcup_{n=1}^{\infty} E_{n}$,则对任意$n$,有$m E_{0}=m\left(E-\bigcup_{n=1}^{\infty} E_{n}\right) \le m\left(E-E_{n}\right)<\frac{1}{n}$;

令$n \rightarrow \infty$,得$m E_{0}=0$。因而可得$E=\left(E-E_{0}\right) \cup E_{0}=\left(\bigcup_{n=1}^{\infty} E_{n}\right) \cup E_{0}=\bigcup_{n=0}^{\infty} E_{n}$。

对任意实数$a$,$E[f>a]=E_{0}[f>a] \cup (\bigcup_{n=1}^{\infty} E_{n}[f>a])$,由$f$在上$E_n$连续,可知$E_{n}[f>a]$可测,而$m^{*}\left(E_{0}[f>a]\right) \leqslant m^{*} E_{0}=0$,所以$E_{0}[f>a]$也可测,从而$E[f>a]$是可测的。

因此$f$是可测的。因为$f$在$E_n$上有限,故在$\bigcup_{n=1}^{\infty} E_{n}$上有限,所以$f(x)$是$a.e.$有限的,证毕。
\end{proof}


\section{本章总结}
在基于之前阐述的猕猴脑科学成像实验和其他脑神经科学实验的结果和启发之后,本章引入了实分析和一些初步的泛函分析的数学工具,构建了基于反馈机制和稀疏编码/解码机制的神经编码模型。类额叶神经编码模型和基于梯度下降算法的深度学习模型最大的不同在于本文的神经元输出的是连续的膜电压,并且并未采用梯度下降而是线性反馈机制的参数更新调整。严谨的数学证明也被给出用于论证设计方法的合理性,并且为了避免抽象的数学表述难以理解,在适当的位置给出了结合现实认知学习原理的形象类比解释。最终,基于提出的学习推理模型,本文还给出了一些有价值的实分析方面的数学通用结论。本章的神经编码模型是将现实实体和初级感知信号转化为压缩编码的底层神经层,在下一章中将基于本章的神经编码模型构造具备学习和推理能力的概念编码模型。

%-------------------------A L G O ------------------------------
\begin{algorithm}[htbp]
\SetAlgoLined
\KwData{编码层和解码层的初始连接权重$w_i,w'_i$;训练集$\mathcal{D}=\{x\}$;}
\KwResult{更新完毕的连接权重$w^{(T)}_i,w'^{(T)}_i$}

initialization\;
\For{$x$ 来自数据集 $\mathcal{D}$}{
    $\hat{x}_i=\xi^i_{\theta}(w_i^{T}x,l)$(神经编码层的第$i$个神经元的输出);
    $x' = w'_i \xi^{-1}_{l_0}(\frac{\partial \hat{x}_i}{\partial l}|_{l=l_0})$(神经解码层的第$i$个神经元的输出);
    $L(x,x')=||x-x'||$(计算反馈的代价信号);
    取$\Delta w,\Delta w' \sim \mathcal{U}(0,m)$(采样更新信号);\\
    \If{有$L(x,x'(w,w'+\Delta w'_{ij}))<L(x,x'(w,w'))$}{$w' = w'+\Delta w'_{ij}$}
    \Else{有$L(x,x'(w,w'+\Delta w'_{ij}))>L(x,x'(w,w'))$}{$w' = w'-\Delta w'_{ij}$}
}
\caption{单个类额叶皮质层渗流神经元的编码解码和参数更新算法流程。}
\label{algo:algo_41}
\end{algorithm}
%-------------------------R I T H M ------------------------------