
%%% Local Variables:
%%% mode: latex
%%% TeX-master: t
%%% End:

\chapter{绪论}
\label{cha:intro}

在本章中将从本文的研究背景情况开始,综述人工智能研究的大背景,并介绍本文要研究的学习和推理问题在人工智能领域对应的位置以及和其他人工智能研究分支方向的联系和区别,特别地,从学科特点的角度简介计算机科学在人工智能研究中的现状及其优势和缺陷,对比生物智能的特点来结合数据科学实验总结对人工智能研究的一些启发即可能的改进(这些可能的改进将会在接下来的章节付诸实现),最终形式化地描述并提出本文将研究的具体问题,总结本文的具体贡献。


  
\section{研究背景与项目情况}
本论文的研究背景是\underline{人工智能}

...

下面将陈述本文的主要贡献,从智能模型的启发、构建、理论研究和应用来讲本文的贡献如下:

\begin{itemize}
  \item \textbf{分析清醒猕猴经视神经双光子成像实验,并给出对构建智能模型有意义结论}:清醒猕猴经视神经双光子成像实验是揭示灵长类动物大脑皮质层关于客观物体进行特征编码原理的高质量实验,根据参与实验和对实验数据的分析,本文给出了很多有意义的关于脑神经皮层编码原理的推测和结论。
  \item \textbf{引入严谨的数学理论构建了智能学习推理系统}: 当前主流人工智能的研究由计算机科学主导,数学严谨性不够,且在效果上也存在限制和缺陷;本文根据实验原理,引入了实分析和泛函分析的理论来构建关于具备智能学习识别能力和推理泛化能力的系统,并在数学理论上证明了其方法的收敛性和优越性。
  \item \textbf{在构建本文数学模型的过程中同时给出了一些有意义的数学结论}: 在引入多个交叉数学分支理论工具来构建关于智能学习推理系统的同时,本文通过推导也得到了一些通用的涉及实分析和泛函分析的有意义的结论并形成定理表述。
  \item \textbf{开发了关于多个应用场景的高度可重用程序}:结合本文的数学理论,在Linux下开发了用于多个应用场景(科学研究、工业控制、文学理解)的高度可重用计算机程序并开源了源代码以验证本文提出的智能系统的可行性,并推广了本文理论的研究成果。
\end{itemize}




\section{论文组织结构}
第一章为绪论部分。主要介绍了本论文的研究背景,介绍了智能学习推理模型的研究意义,还对人类脑认知系统的一些特性进行了简单介绍和分析。最后总结研究的主要问题及本文的主要贡献。 

第二章为专门的综述部分。从人工智能推理模型、发展历史、主要技术方法归类、若干开发问题和研究状况进行了总结,并指出了当前智能学习推理模型的缺陷和改进思路。

第三章介绍了生物实验即脑神经科学实验对智能学习推理模型的启发,探讨了当前人工智能方法和生物机制之间的矛盾。

第四章给出了神经编码系统的数学模型,并从理论方面讨论了模型的性能。

第五章提出了构建智能学习推理模型,并从理论方面讨论了模型的性能,也给出了一些有意义的通用数学结论。

第六章提出了针对科学研究、工业控制、文学理解三个应用场景的智能学习推理模型应用,给出了计算机编程实现,从应用角度证明了本模型的可行性和优越性。

第七章在对全文进行总结的基础上,讨论了本论文工作可能的后续扩展。