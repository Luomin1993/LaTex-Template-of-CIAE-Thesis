
%%% Local Variables: 
%%% mode: latex
%%% TeX-master: t
%%% End: 

\chapter{当前人工智能关于学习推理的研究综述}
\label{cha:china}

\section{智能学习推理的研究方向和典型方法}
虽然当前的人工智能领域取得了令人印象深刻的成果,比如在生成模型、判别模型 \upcite{Kingma2013Auto} \upcite{Goodfellow2014Generative}方面的研究,但是这样的模型都可以认为是对特定的后验概率分布$\mathbb{P}(y|X)$进行采样和拟合,却不具备类脑智能的学习规则和推理泛化的认知能力,另外需要强调的是

...

\item \textbf{迁移学习(Transfer Learning)}:对于训练深度神经网络的迁移学习,就是将一层层网络中每个节点的权重从一个训练好的网络迁移到一个全新的网络里,而不是从头开始为每个特定的任务训练一个神经网络,而训练好的权重可以提取初级特征,之后只需要训练最后几层神经元来完成更高级的认知识别功能,因此这样是高效的。和传统的方法相比,迁移学习的另一个好处其可以做多任务目标的学习,传统的模型面对不同类型的任务,需要训练多个不同的模型,而迁移学习可以先去实现简单的任务,将简单任务中的得到的知识应用到更难的问题上,从而解决标注数据缺少,标注不准确等问题。但是当前的迁移学习模型仍然只能在相似的任务之间迁移,比如相似的场景识别任务之间,距离真正强大的类人泛化迁移能力还有不小的差距,因此构建具备多类型场景迁移学习能力的智能学习推理模型仍然是一个开放性问题。

\item \textbf{因果推断(Causal Inference)}:当前的机器学习几乎完全是统计学或黑箱的形式(即具有不可解释性),从而为其性能带来了严重的理论局限性。这样的系统不能具备因果推断能力,因此不能作为强人工智能的基础 \upcite{Pearl2018Theoretical}。如何将端到端学习与归纳推理相结合,使其对复杂情景具备认知能力逐渐引起了学界的重视,而基于贝叶斯网络的因果推理智能模型给出了一些初步的解决方法 \upcite{Spirtes2010Introduction},但是还不能和类人智能相提并论。因此采用什么样的模型架构,如何实现因果推断能力是至关重要的开放性问题。

\end{itemize}

以上就是对当前智能学习推理开放性问题的描述和总结,这些开放性问题也代表着未来智能学习推理的研究热点和趋势,其中一个重要的趋势仍然是将符号推导系统和模糊数值计算比如神经网络相结合,来构造认知智能更强的学习推理系统,一个典型的研究方向就是图网络(graph network) \upcite{Battaglia2018Relational},这是是对以前各种对图进行操作的神经网络方法的推广和扩展,图网络具有强大的关系归纳偏置,为操纵结构化知识和生成结构化行为提供了一个直接的平台,并定义了一类用于图形结构表示的关系推理的函数,其目标是实现深度学习的因果推理,但是这项研究仍然处于初步的理论阶段,其理论严谨性和可行性还有待考察。但是值得肯定的是将符号系统和连接系统相结合是解决这些开放性问题的一个重要研究趋势。

\section{智能学习推理发展现状总结}
在本章中我们从多个角度综述了当前人工智能关于智能学习推理方向的研究现状,其目的是通过总结,探究智能学习推理的精髓,得出有价值的研究启示。智能学习推理不是一个解耦合的研究方向,而是和深度强化学习,知识图谱信息挖掘,类脑计算密切相关,并且从发展历史的总结,以及对当前主流研究子方向和典型方法的总结上看有两个重要的结论,第一个结论是,当前智能学习推理模型距离类人智能的水平差距还很大,需要调整研究思路;另一个是关于研究趋势的重要思路,即:基于脑神经科学的生物学原理启发,结合基于规则推导的符号系统和基于神经网络的连接系统,就可以兼具推理能力和表征能力,构建更强大的智能学习推理系统。在接下来的部分,我们将根据本章得出的思路,沿着这样的思路开始构建基于额叶皮质层原理启发的具备编码表征能力和概念学习推理能力的数学模型,并给出相应的数学理论讨论和开发计算机程序证明理论的实用性和优越性。