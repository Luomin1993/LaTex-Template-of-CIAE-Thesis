
%%% Local Variables: 
%%% mode: latex
%%% TeX-master: t
%%% End: 

\chapter{智能学习推理系统在数学,工程,文学上的应用}
\label{cha:china}

\section{从应用上对类脑认知能力分类}
本节将阐述如何从智能应用领域的角度对类脑认知能力进行分类


...



\subsection{本文模型在文学信息挖掘任务上的优越性}
本文针对《红楼梦》的信息挖掘任务给出了类额叶概念模型(PFC)的实现方法,通过对实验结果的分析和归因,本文的方法具备如下优点:

\begin{itemize}

\item \textbf{具备零样本学习(zero-shot)的能力}:人物之间的关系概念可以通过计算预测模型$R_{ij} = F^R(v_i) - F^R(v_j)$得出,通过计算$v_R=F^{R^{-1}}(R_{ij})$即可给出抽象概念在词向量空间的编码,值得注意的是$v_R$不必是出现在训练集中的元素,即模型可给出没有接触过的样本结果,因此本文模型具备一定的零样本学习(zero-shot)的能力。

\item \textbf{可学得实体间的关系特征}:除了可以学得实体(也就是类似人物类的名词概念)的编码特征,本文模型也可以给出实体间的关系(即一些谓词特征)编码特征$R_{ij}$,这是得益于模型对特征编码和概念编码功能的解耦合。

\end{itemize}



\section{本章总结}
为了检验本文在前面章节所提出的类额叶神经编码及概念学习泛化模型的可行性,测试其在真实任务中的应用潜力,在本章中分别就数理逻辑智能、工程控制智能、文学语义理解智能三个方向的智能在Linux上结合具体的实验数据集开发了计算机程序验证模型的效果,同时也对比了基线模型的效果。结果表明:基于类脑额叶机制的智能模型在三个任务上都具备较快的收敛率和较佳的最终收敛水平,超过了基于递归/卷积神经网络等纯基于梯度下降算法的学习模型。具体的任务如下:

\begin{itemize}

\item 使用HolStep高阶数理逻辑证明数据集检验智能模型关于逻辑符号代数推理的能力,以卷积神经网络+递归神经网络的组合作为基线模型,以卷积/递归预训练网络作为特征提取器加类脑额叶概念学习模型作为后端学习器的组合解决方案作为本文的模型,最终的实验结果表明将神经网络和类额叶概念学习模型结合会取得更快更好的收敛效果。

\item 使用自主开发的Python程序:核电厂回路工况仿真程序作为物理平台,以递归神经网络(RNN)作为基线模型,将类脑额叶概念学习模型构建为接收实时工况$X_t$并给出下一步操纵参数$\Delta_S$的智能控制模型,最终的实验结果表明以实分析和泛函分析理论为基础的类脑额叶概念学习模型取得了更好的学习效果。

\item 以古典名著《红楼梦》作为训练/测试文本,本文提出的模型分两个层次构成:(1)以Word2Vec方法 \upcite{Mikolov2013Distributed}为基础的词特征向量空间生成算法,将文本符号初步表达为了以词袋/字典为有限集的向量集合;(2)通过类脑额叶概念学习模型,对词特征向量进行再编码,实现更深层次的抽象概念的提取,在再编码映射后的概念空间内可以做求两个小说人物之间概念线性算子的操作,最终再将概念向量投射解码回词空间,即完成了求解人物间抽象关系的计算过程。最终给出了人物概念编码和抽象关系编码的可视化结果。

\end{itemize}