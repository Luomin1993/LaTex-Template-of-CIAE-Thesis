
%%% Local Variables:
%%% mode: latex
%%% TeX-master: t
%%% End:

\chapter{生物实验的启发及其和现阶段主流方法的矛盾}
\label{cha:intro}

在本章中将总结之前在北京大学脑科学研究中心参与的清醒猴脑视觉皮层成像实验及其为构建智能学习推理系统带来的启发,首先介绍实验的基本设定和探究目的,然后对实验结果进行分析,在此之后结合构建智能学习推理系统的目的来讨论生物实验带来的启示,并且给出其数学内涵。在讨论脑科学实验内容后,接下来基于梯度下降的学习算法的核心思维及其和生物实验矛盾之处,从数学的角度证明梯度下降算法的理论缺陷。本章的研究对接下来构建智能学习推理系统的算法具有启发和约束意义。


  
\section{清醒猴视觉皮层成像实验简介:方案及其目的}
清醒猴脑皮层的功能性磁共振成像(fMRI)和双光子成像对于行为神经科学和脑回路机制的研究具有巨大的研究潜力。但是由于各种技术困难, 这一方法暂时只能在初级感知皮层进行,比如视神经皮层、嗅探神经皮层等等。本节讨论的清醒猴脑回路机制的双光子成像的研究方法主要是探究猕猴视觉行为和视觉皮层激活态之间的关系。


\subsection{清醒猴皮层成像实验简介}
脑神经皮层成像的原理是,

...

\section{本章总结}
这一章的总体逻辑上可以分为两部分,首先讨论了基于清醒猴的脑视觉神经皮层的双光子成像实验,得到了许多生物认知行为和生物颅内感知皮层表征之间的有意义的结论,这对于后续构建神经编码数学模型和智能认知推理系统是有启发价值的;第二部分引入了对基于梯度下降算法的深度学习模型的探讨,在这里兼顾了实践性和前沿性地讨论了梯度下降算法的核心思想,以及其在实验上和理论上的缺陷,这对于后续构建本文的学习推理数学模型的优化方法也是有借鉴意义的。


在第一部分的生物实验讨论中,首先介绍了实验的基本设置,原理以及面向的基本目标。接下来就实验的成像结果,给出了稀疏表征编码和解码机制、反馈调节机制是符合本实验结果的结论。再更进一步地,本实验的目的还蕴含了构建类脑智能学习推理系统的目标,相应的本实验的结论也支撑本文给出了构建这样的智能系统的一个初步的数学范式,为后续构建完整的学习推理数学模型打下了基础。


在第二部分对梯度下降算法的讨论中,本文首先抛开其优劣性简明扼要地阐释了梯度下降学习算法的数学内涵和核心思想,接下来从实验和理论方面论证其缺陷,具体来说,其实验缺陷主要表现在和当前前沿的神经生物学和神经解剖学无法统一,因为脑科学研究没有表明梯度下降机制能存在在神经回路中,并且相反的,奖励机制似乎得到了越来越多的生物科学的证据支撑。最后,从数学理论的角度证明了梯度下降算法的一些缺陷。这些讨论有助于后续设计学习算法的代价优化机制。